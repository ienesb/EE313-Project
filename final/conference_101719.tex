\documentclass[conference]{IEEEtran}
\IEEEoverridecommandlockouts
% The preceding line is only needed to identify funding in the first footnote. If that is unneeded, please comment it out.
\usepackage{cite}
\usepackage{amsmath,amssymb,amsfonts}
\usepackage{algorithmic}
\usepackage{graphicx}
\usepackage{textcomp}
\usepackage{xcolor}
\def\BibTeX{{\rm B\kern-.05em{\sc i\kern-.025em b}\kern-.08em
    T\kern-.1667em\lower.7ex\hbox{E}\kern-.125emX}}
\begin{document}

\title{EE313 Project Final Report}

\author{\IEEEauthorblockN{1\textsuperscript{st} Given Name Surname}
\IEEEauthorblockA{\textit{dept. name of organization (of Aff.)} \\
\textit{name of organization (of Aff.)}\\
City, Country \\
email address or ORCID}
\and
\IEEEauthorblockN{2\textsuperscript{nd} Given Name Surname}
\IEEEauthorblockA{\textit{dept. name of organization (of Aff.)} \\
\textit{name of organization (of Aff.)}\\
City, Country \\
email address or ORCID}
\and
}

\maketitle

% \begin{abstract}
% This document is a model and instructions for \LaTeX.
% This and the IEEEtran.cls file define the components of your paper [title, text, heads, etc.]. *CRITICAL: Do Not Use Symbols, Special Characters, Footnotes, 
% or Math in Paper Title or Abstract.
% \end{abstract}

% \begin{IEEEkeywords}
% component, formatting, style, styling, insert
% \end{IEEEkeywords} 

\section{Introduction}

\section{Transmitter}

\subsection{Microphone}
\subsection{Automatic Gain Control}
% ...
\section{Receiver}
\subsection{Photodiode}
\subsection{Low-Pass and High-Pass Filter}
After the light signal is converted to voltage signal, we need to seperate audio signal and reference signal. We need to implement 
a low-pass filter to obtain audio signal and a high-pass filter to obtain reference signal. We decided to use a fourth order Butterworth 
Filter [x] for low-pass filter and second order Butterworth Filter for high-pass filter. The schematics of the filters are shown below. 
%% SCHEMATICs
 
\subsection{peak detector}
At the proposal report, we decided to use a simple circuit with one diode and one capacitor to obtain the amplitude 
of the reference signal and it worked properly at the simulations. However, in practical case, it did not work as we expected 
since the frequency (20kHz) of the signal is too high. Therefore, we changed our design and used the circuit shown in [x] which is 
shown in the figure below. \\
%% SCHEMATIC
The simulation results for the peak detector circuit is shown below.
%% RESULTS

\subsection{Signal Level Indicator}
We are expected to design a circuit to represent received signal level with a single RGB led. Chosen colors for 
each case are shown in the table below.
\begin{table}[htbp]
    \caption{Led colors for each case}
    \begin{center}
    \begin{tabular}{|c|c|c|c|c|}
    \hline
    \textbf{Signal Level} & \textbf{Color}& \textbf{R Pin}& \textbf{G Pin}& \textbf{B Pin} \\
    \hline
    No Signal & - & 0 & 0 & 0\\
    \hline
    Weak Signal & Red & 1 & 0 & 0\\
    \hline
    Moderate Signal & Yellow & 1 & 1 & 0\\
    \hline
    Good Signal & Green & 0 & 1 & 0\\
    \hline
    No Signal & Blue & 0 & 0 & 1\\
    \hline
    \end{tabular}
    \label{tab1}
    \end{center}
\end{table}
The overall schematic of the signal level indicator circuit is shown below. \\
% SCHEMATIC
At first, we used 5 resistors connected in serial between Vcc and ground to divide Vcc into four different voltages 
to determine regions for each signal level case. 
Then four comparator is used to determine at which region the amplitude of the reference signal is. When the amplitude 
is at the first region, all comparators will have negative output and no color will be displayed. When the amplitude 
is at the second and third region, the first and the second comparator will have a positive output and red and yellow 
color will be displayed respectively. The output of the third comparator is substracted from the output of the first 
comparator by the difference amplifier so that the red pin will be turned off when the signal is at the fourth region 
and green color will be displayed. Similarly, the output of the last comparator is substracted from the output of the 
second comparator so that the green pin will be turned off and blue color will be displayed when the amplitude is at the 
last region. \\

We used common anode RGB led so that the led has a common ground pin. We used different resistors for each color pin of the led 
to adjust the tone of the colors. \\ 

The simulation results are shown below.\\
%% RESULTS

\subsection{audio amplifier}


\section{Conclusion}

\section*{References}
% peak detector
% https://www.analog.com/en/technical-articles/ltc6244-high-speed-peak-detector.html
\end{document}
