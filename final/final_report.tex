\documentclass[conference]{IEEEtran}
\IEEEoverridecommandlockouts
% The preceding line is only needed to identify funding in the first footnote. If that is unneeded, please comment it out.
\usepackage{cite}
\usepackage{amsmath,amssymb,amsfonts}
\usepackage{algorithmic}
\usepackage{graphicx}
\usepackage{textcomp}
\usepackage{xcolor}
\def\BibTeX{{\rm B\kern-.05em{\sc i\kern-.025em b}\kern-.08em
    T\kern-.1667em\lower.7ex\hbox{E}\kern-.125emX}}
\begin{document}

\title{EE313 Project Final Report}

\author{\IEEEauthorblockN{Ahmet Caner Akar}
\IEEEauthorblockA{\textit{Electrical and Electronics Engineering Department} \\
\textit{Middle East Technical University}\\
Ankara, Turkey \\
e244228@metu.edu.tr}
\and
\IEEEauthorblockN{İsmail Enes Bülbül}
\IEEEauthorblockA{\textit{Electrical and Electronics Engineering Department} \\
\textit{Middle East Technical University}\\
Ankara, Turkey \\
e244263@metu.edu.tr}
\and
}

\maketitle

\begin{abstract}
This document is about the end-term project of EE313 Analog Electronics Laboratory, namely design of an optical wireless communication system: photophone. Background theoretical knowledge, literature research and various work about design methods and mathematical analysis of them related to this project together with simulation and experimental results are defined in this document. 
\end{abstract}

% \begin{IEEEkeywords}
% component, formatting, style, styling, insert
% \end{IEEEkeywords} 

\section{Introduction}
Communication is an integral part of our lives for us humans, who are social beings. While this communication was carried out by methods such as pigeons or fire in history, various types of communication have emerged with the advancement of technology. In this project, we will examine the one of the modern communication systems: optical wireless communication system. The overall diagram of the project is given in Figure 1, below for better understanding. \\ \\
The aim of the project is to transmit the audio input signal that is generated by the microphone and to receive this information wirelessly. Then, the received signal is fed to the speaker at the final step while the quality of the signal is indicated by a single RGB LED. In general, the project can be grouped under two main units: Transmitter Unit and Receiver Unit, as it can be seen in Figure 1. Also, each main part consists of different sub-units, and they are explained in detail in the following sections of the report.
\section{Rules}
Maximum allowed DC Voltage: ±15 Volts. \\
Instruments not allowed to be used: 6V terminal of the DC supply.\\ 
Frequency Range for Reference Signal: 10 kHz – 30 kHz. \\
Component not allowed to be used: audio op-amps, microphone with integrated driving circuitries, infrared and ultraviolet lasers, and visible light lasers whose power > 5mW.
\section{Transmitter}

\subsection{Microphone Driver}
\subsection{Automatic Gain Control}
% ...
\section{Receiver}
\subsection{Photodiode}
\subsection{LPF}
\subsection{HPF}
\subsection{peak detector}
\subsection{comparator}
\subsection{audio amplifier}


\section{Conclusion}

\section*{References}

\end{document}
