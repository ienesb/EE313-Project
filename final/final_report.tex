\documentclass[conference]{IEEEtran}
\IEEEoverridecommandlockouts
% The preceding line is only needed to identify funding in the first footnote. If that is unneeded, please comment it out.
\usepackage{cite}
\usepackage{amsmath,amssymb,amsfonts}
\usepackage{algorithmic}
\usepackage{graphicx}
\usepackage{textcomp}
\usepackage{xcolor}
\def\BibTeX{{\rm B\kern-.05em{\sc i\kern-.025em b}\kern-.08em
    T\kern-.1667em\lower.7ex\hbox{E}\kern-.125emX}}
\begin{document}

\title{EE313 Project Final Report}

\author{\IEEEauthorblockN{Ahmet Caner Akar}
\IEEEauthorblockA{\textit{Electrical and Electronics Engineering Department} \\
\textit{Middle East Technical University}\\
Ankara, Turkey \\
e244228@metu.edu.tr}
\and
\IEEEauthorblockN{İsmail Enes Bülbül}
\IEEEauthorblockA{\textit{Electrical and Electronics Engineering Department} \\
\textit{Middle East Technical University}\\
Ankara, Turkey \\
e244263@metu.edu.tr}
\and
}

\maketitle

\begin{abstract}
This document is about the end-term project of EE313 Analog Electronics Laboratory, namely design of an optical wireless communication system: photophone. Background theoretical knowledge, literature research and various work about design methods and mathematical analysis of them related to this project together with simulation and experimental results are defined in this document. 
\end{abstract}

% \begin{IEEEkeywords}
% component, formatting, style, styling, insert
% \end{IEEEkeywords} 

\section{Introduction}
Communication is an integral part of our lives for us humans, who are social beings. While this communication was carried out by methods such as pigeons or fire in history, various types of communication have emerged with the advancement of technology. In this project, we will examine the one of the modern communication systems: optical wireless communication system. The overall diagram of the project is given in Figure 1, below for better understanding.\\ \par
The aim of the project is to transmit the audio input signal that is generated by the microphone and to receive this information wirelessly. Then, the received signal is fed to the speaker at the final step while the quality of the signal is indicated by a single RGB LED. In general, the project can be grouped under two main units: Transmitter Unit and Receiver Unit, as it can be seen in Figure 1. Also, each main part consists of different sub-units, and they are explained in detail in the following sections of the report.
\section{Rules}
\noindent Maximum allowed DC Voltage: ±15 Volts. \\
Instruments not allowed using: 6V terminal of the DC supply.\\ 
Frequency Range for Reference Signal: 10 kHz – 30 kHz. \\
Component not allowed to be used: audio op-amps, microphone with integrated driving circuitries, infrared and ultraviolet lasers, and visible light lasers whose power \(>\) 5mW.
\section{Transmitter Unit}
\subsection{Microphone Driver}
The first part of the transmitter unit is microphone driver circuit. To transmit an audio signal using a laser, we need to detect this audio signal first. Therefore, to do this we used an electret microphone. It requires a biasing voltage to operate. Thus, we biased the microphone by connecting the positive terminal of it to the 1 k\(\Omega\) resistor. However, since the output voltage of the microphone is quite low, we cannot directly connect it to the rest of the circuit. In order to use this output, first, we should amplify it with a non-inverting amplifier circuit as shown in Figure 2. \\ \par
There is a 10 k\(\Omega\) potentiometer connected between ground and the inverting input of the amplifier so that by changing its value, the gain can be adjusted, and the amplitude of the output signal is changed. The gain of the topology can be found by the following expression, Gain = (R\(_3\)+R\(_4\))/R\(_4\). 
Also, the simulation result of the input-output characteristics of the microphone driving circuit in LTspice is given in Figure 3 when R\(_4\) = 10k\(\Omega\). \\ \par
After non-inverting amplifier circuit, we connected a buffer circuit so that the microphone driver will not be affected from the rest of the circuit. 
\subsection{Automatic Gain Control}
The second sub-unit of the transmitter part is Automatic Gain Control (AGC). We should adjust the output signal of the microphone driver circuit because the output of the microphone is distance and frequency dependent, so the output amplitude of the microphone change with time as well as distance of the speaker (person) to it. Therefore, as it is stated in project definition, we need an automatic gain controller that controls gain and adjusts the amplitude of the microphone signal so that we will get a relatively constant amplitude audio signal at the output of the AGC regardless of the amplitude of the input signal. 
\section{Receiver Unit}
\subsection{Photodiode}
\subsection{LPF}
\subsection{HPF}
\subsection{peak detector}
\subsection{comparator}
\subsection{audio amplifier}


\section{Conclusion}

\section*{References}

\end{document}
