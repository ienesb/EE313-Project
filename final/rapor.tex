\documentclass[12pt]{article}
\usepackage[utf8]{inputenc, }
\usepackage{graphicx}
\usepackage{svg}
\usepackage{hyperref}
\usepackage[margin=1in]{geometry}
\usepackage{setspace}
\usepackage{color}
\usepackage{pdfpages}
\usepackage{amsmath}
\usepackage{float}

\sloppy
\definecolor{lightgray}{gray}{0.5}
\setlength{\parindent}{0pt}

\hypersetup{
    colorlinks,
    citecolor=black,
    filecolor=black,
    linkcolor=black,
    urlcolor=black
    pdftitle={EE300 İsmail Enes Bülbül}
}
\onehalfspacing



% \title{EE230 Homework-1}
\author{İsmail Enes Bülbül}
\date{October 2022}
\renewcommand*\contentsname{Table of Contents}
\renewcommand*{\refname}{}
\begin{document}
% \maketitle
% \tableofcontents
\newpage

\begin{titlepage}
    \begin{center}


        % \vspace{1cm}
        \includegraphics[width=0.7\textwidth]{logo.jpg}  \\
        % \vspace{1cm}  
        \Huge
        \textbf{MIDDLE EAST TECHNICAL UNIVERSITY}
            
            
        \vspace{1.5cm}
        \LARGE    
        \textbf{ELECTRICAL - ELECTRONICS ENGINEERING}
        \vspace{1.5cm}
            
        \textbf{EE300 SUMMER PRACTICE REPORT}
            
        \vfill
            
            
        \vspace{0.8cm}
            
        
            
        \normalsize
        \begin{flushleft}        
        \textbf{Student Name:} İsmail Enes Bülbül \\
        \textbf{Student ID:} 2442630\\
        \textbf{SP Company:} InterLabs İleri Ar-Ge Uzay Havacılık ve Bilişim Teknolojileri A.Ş.\\
        \textbf{SP Date:} 05.09.2022-30.09.2022\\ 
        \textbf{Supervisor Name:} Barış Canatan\\ 
        \textbf{Supervisor Email:} baris.canatan@interlabs.com.tr\\ 
        \textbf{Supervisor Signature:} \\    
        \end{flushleft}

    \end{center}
\end{titlepage}

    \tableofcontents
    \newpage

    \section{Introduction}
    
    I started my summer practice on 05.09.2022 and completed it on 
    30.09.2022 in InterLabs İleri Ar-Ge Uzay Havacılık ve 
    Bilişim Teknolojileri A.Ş. in Ankara. I worked in Signal Processing 
    department and my supervisor was Barış Canatan who is an electronics 
    engineer working in this department. \\
    In the first week of my summer practice, I studied Kalman Filter and EKF. 
    I did some exercises to gain intuition about Kalman Filter. 
    In the second week, I implemented a method to estimate the distance from 
    RSSI measurements by using Kalman Filter. In the last two weeks, I used EKF to 
    estimate orientation and position by using inertial measurements.

    \section{Description of the Company}

    \subsection{Company Name}
    InterLabs İleri Ar-Ge Uzay Havacılık ve Bilişim Teknolojileri A.Ş.

    \subsection{Company Location}
    \textbf{Address:} Söğütözü Caddesi, Koç Kuleleri A Blok, No:47, 
    Çankaya-Ankara/Türkiye\\
    \textbf{Phone:} +90 312 225 10 93

    \subsection{General Description of the Company}
    InterLabs was the R\&D department of InterProbe corporation, and was established 
    as a spin-off company in R\&D. InterProbe focused on cyber security and cryptography 
    in Pavo Group. \\
    Interlabs has 40 employees of which 25 are electronics engineers. The company has 4 
    departments which are Signal Processing, Computer Vision, UAV Systems, and 
    Software Development. \\
    \textbf{Signal Processing:} The department is currently focused on 
    Signals of Opportunity (SoOp) systems. SoOp can be defined as positioning 
    methods without using GPS signals.\\
    \textbf{Computer Vision:} The department is focused on developing 
    face-recognition algorithms for CCTVs.\\
    \textbf{UAV Systems:} The department is focused on developing swarm 
    algorithms for UAV systems.\\
    \textbf{Software Development:} The department works with other departments 
    to meet their programming requirements. 

    \section{First Week}    
    In the first week, I study Kalman Filter since it is a commonly used 
    algorithm in control applications. I also did some examples to apply 
    Kalman Filter and gain intuition.

    \subsection{Kalman Filter}
    Kalman Filter is used to reduce the noise of the measurements and 
    combine several measurements. We can also use it to estimate some 
    states which we cannot measure directly. 
    Kalman Filter has two phases which are prediction and measurement. 
    In the prediction phase, we try to predict the state in the future by using 
    the current estimates. We can model this phase as (1).

    \begin{equation}\label{eq:1}
        x_{k+1} = Fx_k + w_k
    \end{equation}
    In (1), \(x_k\) represents the state vector at time \(k\), \(F\) 
    represents the transition matrix which is used to predict state vector 
    at time \(k+1\), and \(w\) is zero-mean gaussian noise which represents 
    uncertainties of the system. \\
    In the measurement phase, we use both predictions and measurements to 
    obtain a better estimate. We can model this as (2).
    \begin{equation}\label{eq:2}
        z_k = Hx_k + v_k
    \end{equation}
    In (2), \(z_k\) is the measurement vector, \(H\) is the observation matrix,
    and \(v_k\) is a zero-mean gaussian noise which represents uncertainties 
    in the system. \\
    In both phases, we also try to determine the covariance matrices of 
    the state variables. Lastly, we need to calculate Kalman gain. 
    Kalman gain is used in the measurement phase to determine the balance 
    between measurements and predictions. All 5 equations used in Kalman 
    Filter can be derived as (3-7) \cite{kalman}.
    \begin{equation}\label{eq:3}
        x^{p}_{k+1} = Fx^{e}_k
    \end{equation}
    \begin{equation}\label{eq:4}
        P^{p}_{k} = FP^{e}_{k-1}F^T + Q_{k-1}
    \end{equation}
    \begin{equation}\label{eq:5}
        K_k = P^p_kH^T(HP^p_kH^T + R_k)^{-1}	
    \end{equation}
    \begin{equation}\label{eq:6}
        x^{e}_{k} = x^{p}_k + K_k(z_k - Hx^p_k)	
    \end{equation}
    \begin{equation}\label{eq:7}
        P^{e}_{k} = (I - K_kH)P^p_k	
    \end{equation}
    In (3-7), subscripts represent time, superscripts represent whether the 
    variable is estimated or predicted, \(P\) represents the covariance matrix 
    of \(x\), \(Q\) and \(R\) represent the covariance matrices of \(w_k\) and 
    \(v_k\) respectively, and \(K\) represents Kalman gain. \\
    First, (3-4) is used to predict the state vector and its covariance matrix. 
    Then (5) is used to calculate Kalman gain. Lastly, (6-7) is used to 
    estimate the state vector and its covariance matrix by using measurements. 

    \subsection{Extended Kalman Filter (EKF)}
    EKF is another version of Kalman Filter where the prediction and 
    measurement phases have nonlinear functions. These phases can 
    be modeled as (8-9).
    \begin{equation}\label{eq:8}
        x_{k+1} = f(x_k) + w_k
    \end{equation}
    \begin{equation}\label{eq:9}
        z_k = h(x_k) + v_k
    \end{equation}
    We can use these nonlinear functions to predict and estimate 
    the state variables. However, we cannot determine the covariance 
    matrices and Kalman gain. To do this, we need to calculate their 
    jacobian matrices to linearize these functions. As a result, 
    we can derive EKF equations as (10-14) \cite{ekf}.
    \begin{equation}\label{eq:10}
        x^{p}_{k+1} = f(x^{e}_k)
    \end{equation}
    \begin{equation}\label{eq:11}
        P^{p}_{k} = J_fP^{e}_{k-1}J_f^T + Q_{k-1}
    \end{equation}
    \begin{equation}\label{eq:12}
        K_k = P^p_kJ_h^T(J_hP^p_kJ_h^T + R_k)^{-1}	
    \end{equation}
    \begin{equation}\label{eq:13}
        x^{e}_{k} = x^{p}_k + K_k(z_k - h(x^p_k))	
    \end{equation}
    \begin{equation}\label{eq:14}
        P^{e}_{k} = (I - K_kJ_h)P^p_k	
    \end{equation}
    In (11,12,14), \(J_f\) represents the first jacobian of \(f\) with 
    respect to thestate vector and \(J_h\) represents the first jacobian 
    of \(h\) with respect to the state vector.

    \section{Second Week}
    In the second week, I researched distance estimation methods using 
    Received Signal Strength Indication (RSSI) measurement of a 
    received signal. RSSI measurement is related to the power of the 
    received signal. The relation between the RSSI value and the distance 
    between the source of the signal and the receiver can be defined 
    as (15) \cite{logmodel}.
    \begin{equation}\label{eq:15}
        RSSI = 10nlog_{10}d + A
    \end{equation}
    In (15), \(d\) represents the distance, and \(n\) and \(A\) are tuned to 
    fit the environment. Although we can obtain a distance estimation by 
    using (15), it can be inaccurate due to multipath effect. 
    Multipath effect can be defined as receiving one signal by several 
    paths due to reflection or refraction. This can make the received signals 
    attenuated or amplificated unexpectedly. To reduce multipath effect 
    and obtain a better distance estimation, I tried to implement the method 
    proposed in \cite{shue} that uses Kalman Filter.

    \subsection{Method Description}
    The state vector consists of the distance and velocity of the receiver. 
    To predict the state vector at time \(k+1\), (16,17) is used.
    \begin{equation}\label{eq:16}
        d_{k+1} = d_{k} + \Delta t v_{k}
    \end{equation}
    \begin{equation}\label{eq:17}
        v_{k+1} = v_{k}
    \end{equation}
    In (16), the Euler method is used to predict distance, and in (17), velocity 
    is assumed to be constant. \\
    The measurement vector consists of RSSI measurements. In the measurement phase, 
    the measurement is converted to distance by using (15).\\
    F and H matrices can be calculated as (18).
    \begin{equation}\label{eq:18}
        F = 
        \begin{bmatrix}
            1 & \Delta t\\
            0 & 1
        \end{bmatrix}\;\;\;
        H = 
        \begin{bmatrix}
            1 & 0
        \end{bmatrix}
    \end{equation}
    \subsection{Simulations}

    \subsubsection{Simulation Environment}
    In order to test the method, I generated some RSSI data in MATLAB. 
    To do so, I adjusted n to 3 and A to 30 in (15). 
    Then I added zero-mean gaussian noise to the generated RSSI values. I also 
    added gaussian noises with non-zero mean at some points. 
    The graph of RSSI vs Distances of the generated dataset is plotted 
    in Fig. 1.
    \begin{figure}[H]
        \centerline{\includesvg[inkscapelatex=false, scale=0.6]{rssivsdistance}}
        \caption{The graph of generated RSSI data}
    \end{figure}

    \subsubsection{Simulation Results}
    I implemented the method by using the dataset generated. For comparison, 
    I also used (15) to obtain distance estimations. The results are 
    plotted in Fig. 2.
    \begin{figure}[H]
        \centerline{\includesvg[inkscapelatex=false, scale=0.6]{rssiResults}}
        \caption{The results of simulation}
    \end{figure}
    From Fig. 2, we can see that by using Kalman Filter, we can reduce 
    the multipath effect and obtain a better distance estimation by using 
    the RSSI measurements.

    \section{Third and Fourth Week}   
    In the rest of my summer practice, I researched orientation and position 
    estimation methods using inertial measurements only. By doing this, I 
    learned how IMU measurements are modeled, 
    orientation representation methods, and coordinate systems.
    \subsection{Inertial Measurement Unit (IMU)}
    IMU consists of three units which are accelerometer, gyroscope, and 
    magnetometer. 

    \subsubsection{Accelerometer}
    Accelerometer measures the acceleration of the device in the body frame. 
    It measures acceleration by measuring the force acting on the device. 
    Therefore, the measurement involves gravitational acceleration whether 
    the device is falling freely or not. The measurement can be modeled as (19) \cite{kok}.
    \begin{equation}\label{eq:19}
        y_{accelerometer} = a_{body} - g_{body} + w
    \end{equation}
    In (19), \(y\) is the reading of the accelerometer, \(a\) is the acceleration of the device 
    in the body frame, \(g\) is the gravitational acceleration in the body frame 
    and \(w\) represents the noise in the measurement.

    \subsubsection{Gyroscope}
    Gyroscope measures the angular velocity of the device. The measurement involves 
    the earth's rotation. However, it is too small compared to the angular velocity 
    of the device with respect to the earth. Therefore, we can neglect the earth's 
    rotation. The measurement can be modeled as (20) \cite{kok}.
    \begin{equation}\label{eq:20}
        y_{gyroscope} = \omega_{body} + w
    \end{equation}
    In (20), \(y\) is the reading of the gyroscope, \(\omega\) is the angular velocity of the 
    device in the body frame and \(w\) represents the noise in the measurement.

    \subsubsection{Magnetometer}
    Magnetometer measures the magnetic field in the body frame. We assume that the 
    magnetometer is calibrated, therefore it can be used to measure the earth's 
    magnetic field. The measurement can be modeled as (21) \cite{kok}.
    \begin{equation}\label{eq:21}
        y_{magnetometer} = m_{body} + w
    \end{equation}
    In (21), \(y\) is the reading of the magnetometer, \(m\) is the magnetic field of the 
    earth in the body frame and \(w\) represents the noise in the measurement.
      
    \subsection{Rotations}

    \subsubsection{Euler Angles}
    Euler angles involves three angles which are used for three individual 
    rotations around axes. To apply a rotation represented with 
    Euler angles, first of all, the frame is rotated around the first axis 
    by the first angle. Then, the frame is rotated around the second axis 
    of the rotated frame by the second angle, and so on. Although Euler angles 
    is an intuitive way to represent rotations, it may have singularity problem 
    in some cases \cite{quat}.

    \subsubsection{Rotation matrix}
    Rotation matrices are 3x3 matrices that can be used to rotate vectors by 
    multiplying them from left. Rotation matrix can be calculated from Euler angles 
    in ZYX order (\(\psi\theta\phi\)) by using (22) \cite{quat}. % \psi \theta \phi
    \begin{equation}\label{eq:22}
        R = \begin{bmatrix}
            cos\psi cos\theta\;\; cos\psi sin\theta sin\phi - sin\psi cos\phi\;\; cos\phi sin\theta cos\phi + sin\phi sin\phi\\
            sin\psi cos\theta\;\; sin\psi sin\theta sin\phi + cos\psi cos\phi\;\; sin\psi sin\theta cos\phi - cos\psi sin\phi\\   
            -sin\theta\;\; cos\theta sin\phi\;\; cos\theta cos\phi\\    
            
        \end{bmatrix}
    \end{equation} 

    \subsubsection{Quaternions}
    Quaternion is an extended version of complex numbers. It involves one real 
    scalar and three imaginary parts. To represents 3x1 vectors, we assign 
    each element of the vector to imaginary parts and make the real part 0. 
    The orientations can be represented as quaternions with non-zero real parts. 
    To rotate a vector \(v\), we can use quaternion multiplication in (23).
    \begin{equation}\label{eq:23}
        v_{rotated} = q*v*\bar{q}
    \end{equation}     
    Although it is not an intuitive way to represent rotations, 
    it is commonly used since it solves the singularity problem in Euler angles. 
    In the rest of my summer practice, I generally used quaternions to represent the 
    orientation of a vehicle \cite{quat}.

    \subsection{Coordinate Systems}

    \subsubsection{ECEF (Earth-centered, Earth-fixed)}
    ECEF is a stationary frame with respect to the earth. The origin of the 
    frame is the center of the earth. The X and Y axis are in the same plane 
    with the equator. X axis passes through \(0^{\circ}\) and \(180^{\circ}\) longitude, 
    and Y axis passes through \(90^{\circ}\) E and \(90^{\circ}\) W longitudes. 
    Z axis passes through the poles.
    
    \subsubsection{LLA (Latitude Longitude Altitude)}
    We can use the latitude, longitude, and altitude of a vehicle to represent 
    its position in geographic coordinate system.
    
    \subsubsection{NED (North East Down)}
    NED takes a point on earth as its origin. North, East, and Down are the axes 
    of the frame in ordered way.

    \subsubsection{Body frame}
    Body frame takes the center of the vehicle or IMU as its origin. The three axes 
    point to the front, right and down of the vehicle respectively.

    \subsection{Method Description}
    I decided to implement the model described in \cite{kok} to estimate orientation. 
    It uses an EKF to estimate orientation. The state vector contains 
    quaternions that rotates a point in the body frame to NED frame. The 
    measurement vector contains accelerometer and magnetometer readings. 
    The gyroscope readings are used as the input vector to predict future 
    quaternions by calculating the derivative of quaternions. Derivatives of 
    quaternions are calculated by using (24).
    \begin{equation}\label{eq:24}
        \frac{dq}{dt} = \frac{1}{2}q*\omega_{body}
    \end{equation}
    To predict future quaternions, (25) is used.
    \begin{equation}\label{eq:25}
        q_{k+1} = q_{k} + \Delta t\frac{dq_k}{dt}
    \end{equation}
    To estimate quaternions by using predicted quaternions and accelerometer 
    and magnetometer readings, (26) and (27) are used.
    \begin{equation}\label{eq:26}
        [0\; 0\; 9.8]^T = q_k*y_{accelerometer}*\bar{q_k}
    \end{equation}
    \begin{equation}\label{eq:27}
        m_{NED} = q_k*y_{magnetometer}*\bar{q_k}
    \end{equation}
    In (26), it is assumed that the acceleration of the vehicle is 
    too small compared to the gravitational acceleration. Therefore, 
    we can use accelerometer to measure gravitational acceleration in the body frame, 
    and quaternions are used to rotate the reading to NED frame. 
    In the NED frame, the accelerometer reading corresponds to gravitational 
    acceleration which is \([0\; 0\; 9.8]^T m/s^2\). \\
    In (27), the magnetometer reading is rotated to NED frame. 
    It corresponds to the magnetic field of the earth in NED frame which 
    can be obtained by using the model described in \cite{vmm}. \\
    By using (25), (26), and (27), I decided on f and h functions and 
    calculated their jacobian matrices.\\
    After estimating the orientation, I used it to estimate the position. 
    To estimate the position and the velocity, I used the Euler equations 
    which are stated in (28) and (29).
    \begin{equation}\label{eq:28}
        p_{k+1} = p_k + \Delta tv_k + 1/2\Delta t^2a_k
    \end{equation}
    \begin{equation}\label{eq:29}
        v_{k+1} = v_k + \Delta ta_k
    \end{equation}
    To obtain the acceleration in ECEF frame, first of all, I rotated 
    the accelerometer reading from the body frame to NED frame. Then, 
    I simply used ned2ecefv function of MATLAB to obtain acceleration 
    in ECEF frame.

    \subsection{Simulations}

    \subsubsection{Simulation Environment}
    To test the model, I created a trajectory by using waypointTrajectory 
    object of MATLAB for a fixed-wing plane. From this trajectory, I obtained 
    truth values of position, orientation, and other pose properties. The trajectory 
    showed in Fig. 3.\\
    % figure geoscatter, yaw-pitch-roll, positionNed

    \begin{figure}[H]
        \centerline{\includesvg[inkscapelatex=false, scale=0.6]{5.1geo}}
        \caption{The generated trajectory in map.}
    \end{figure}
    The graphs of orientation and position of the plane are plotted in Fig. 4 and 5.
    \begin{figure}[H]
        \centerline{\includesvg[inkscapelatex=false, scale=0.6]{5.1orient}}
        \caption{The graph of orientation of plane.}
    \end{figure}
    \begin{figure}[H]
        \centerline{\includesvg[inkscapelatex=false, scale=0.6]{5.1position}}
        \caption{The graph of position of plane.}
    \end{figure}
    
    For these truth values, I obtained IMU readings by using imuSensor 
    object of MATLAB. I 
    adjusted IMU parameters as indicated in the datasheet of the product 
    that the company will use. The graphs of generated data of IMU 
    readings by time are plotted in Fig. 6, 7 and 8.
    % figure y_acc y_gyro y_mag

    \begin{figure}[H]
        \centerline{\includesvg[inkscapelatex=false, scale=0.6]{5.2acc.svg}}
        \caption{The graph of acceleration readings of IMU.}
    \end{figure}
    \begin{figure}[H]
        \centerline{\includesvg[inkscapelatex=false, scale=0.6]{5.2gyro.svg}}
        \caption{The graph of gyroscope readings of IMU.}
    \end{figure}
    \begin{figure}[H]
        \centerline{\includesvg[inkscapelatex=false, scale=0.6]{5.2mag.svg}}
        \caption{The graph of magnetometer readings of IMU.}
    \end{figure}

    
    \subsubsection{Simulation Results}
    To examine the results easily, I converted estimated quaternions 
    to Euler angles in ZYX order (yaw-pitch-roll) and position estimations 
    in ECEF frame to NED frame. The error in orientation estimation is 
    plotted in Fig. 9. \\
    % figure angular distance, yaw-pitch-roll

    \begin{figure}[H]
        \centerline{\includesvg[inkscapelatex=false, scale=0.6]{5.3distance.svg}}
        \caption{The graph of error in orientation estimation.}
    \end{figure}
    The estimated orientations in Euler angles are plotted in Fig. 10, 11, and 12.
    \begin{figure}[H]
        \centerline{\includesvg[inkscapelatex=false, scale=0.6]{5.3yaw.svg}}
        \caption{The graph of yaw angle estimations.}
    \end{figure}
    \begin{figure}[H]
        \centerline{\includesvg[inkscapelatex=false, scale=0.6]{5.3pitch.svg}}
        \caption{The graph of pitch angle estimations.}
    \end{figure}
    \begin{figure}[H]
        \centerline{\includesvg[inkscapelatex=false, scale=0.6]{5.3roll.svg}}
        \caption{The graph of roll angle estimations.}
    \end{figure}

    From the graphs, we can see that the EKF estimates the orientation 
    with an error of less than 2.5 degrees. The error in orientation estimation 
    increases when the acceleration of the vehicle is not too small as 
    it violates the assumption in (26).\\
    The position estimation is done by only integrating accelerometer 
    readings. The estimations are not corrected by using any equation, 
    therefore, we have a divergent error. At the point where the plane 
    rotates to right, the orientation error is increased since the acceleration 
    of the plane is increased which violates the assumption in (26).
    The fact that we could not estimate orientation accurately also increases 
    the error in position estimations since orientation estimations are used to rotate 
    accelerometer readings and remove gravitational acceleration.\\
    The estimated positions are shown in Fig. 13.
    % figure geoscatter, norm(distance)

    \begin{figure}[H]
        \centerline{\includesvg[inkscapelatex=false, scale=0.6]{5.4geo.svg}}
        \caption{The estimated positions in map.}
    \end{figure}
    The error in position estimations is plotted in Fig. 14.
    \begin{figure}[H]
        \centerline{\includesvg[inkscapelatex=false, scale=0.6]{5.4norm.svg}}
        \caption{The graph of error in position estimation.}
    \end{figure}

    \section{Conclusion}   
    I completed my summer practice at InterLabs in Ankara. I learned the math behind Kalman Filter 
    and EKF. I implemented some models based on EKF. I have done many 
    practices on MATLAB. I have also learned mathematical models to represent 
    rotations and coordinate systems.\\ 
    During my summer practice, I experienced how research projects are done. I also observed 
    the development process of projects. I recommend the company 
    to one who wants to learn how R\&D projects are maintained in the private sector.
    
    % I recommend the company for summer practice since the 
    % engineers working there are kind and try to help interns to improve
    % themselves.   
    \newpage
    \section{References}   
    \bibliographystyle{ieeetr}
    \bibliography{refs}

    % rssi shue
    % quaternion
    % ekf ori
    % world model
    % ekf tutorial

\end{document}